\documentclass{beamer}
\mode<presentation> {

  \usetheme{Boadilla}
  \usecolortheme{dolphin}
  % \setbeamertemplate{footline}
 \setbeamercovered{transparent}
}

\usepackage[german]{babel}
\usepackage[style=authoryear, backend=bibtex]{biblatex} 
   \bibliography{bibliography} 
\usepackage{eurosym}
\title{Irrationalit\"at}
   
\author{Stefan Gabler \& Daniel Winkler}

\date{3. Mai 2017}

\begin{document}

\begin{frame}[plain]
  \titlepage
  \textcite{oliver2003quantitative}\\\citetitle{oliver2003quantitative}\\

\end{frame}

\begin{frame}
  \frametitle{\"Uberblick}
  \tableofcontents
\end{frame}

\section{Experiment}
\label{sec:experiment}

\begin{frame}  
 \frametitle{Rationalit\"at unter VWL-Studierenden}
  
 % \begin{columns}
    
 %   \begin{column}{0.5\textwidth}
      
  \begin{block}{1. Entscheidung}
   \begin{itemize}
      \item[A] Sie bekommen \euro 100  mit Sicherheit.
      \item[B] Sie haben eine 10\% Chance auf \euro 500,\\ eine 89\% Chance auf \euro 100\\ und eine 1\% Chance nichts zu bekommen.
     \end{itemize}

   \end{block}

 %  \end{column}      

 % \begin{column}{0.5\textwidth}

    \begin{block}{2. Entscheidung}
      \begin{itemize}
      \item[C] Sie haben eine 11\% Chance auf \euro 100\\und eine 89\% Chance nichts zu gewinnen.
      \item[D] Sie haben eine 10\% Chance auf \euro 500\\und eine 90\% Chance nichts zu gewinnen.
      \end{itemize}
   \end{block}

 %  \end{column}
  
 % \end{columns}
  
 \end{frame}



\section{Axiome der Erwartungsnutzentheorie}
\label{sec:einfuhrung}
\begin{frame}
  \frametitle{Die Axiome nach Von Neumann \& Morgenstern}
   \begin{enumerate}
  \item<1,4> Unabh"angigkeit von (irrelevanten) Alternativen.\\
    F"ur alle $x, y, z \in$ A\footnote{Menge der Alternative} und alle $\alpha \in (0,1)$:\\
    \begin{equation}
      \label{eq:4}
      x \succ y \Rightarrow \alpha x +(1-\alpha) z \succ \alpha y + (1-\alpha) z  \tag{Unabh"angigkeit}
    \end{equation}

        
  \item<2> F"ur alle $x, y, z \in$ A:
      \begin{equation}
      \label{eq:1}
     x \succ y \succ z \Rightarrow x \succ z \tag{Transitivit"at}
    \end{equation}
    F\"ur alle x, y $\in$ A gilt eine und nur eine der folgenden Relationen:
    \begin{equation}
      \label{eq:2}
     x \succ y \text{ oder } y \succ x \text{ oder } x \sim y \tag{Vollst"andigkeit  \& Asymmetrie}
    \end{equation}
  \item<3> Es existieren ein $\alpha \in [0,1]$, sodass wenn $x \succ y \succ z$:
    \begin{equation}
      \label{eq:3}
      y \sim \alpha x + (1-\alpha) z \tag{Stetigkeit}
    \end{equation}
  \end{enumerate}

\end{frame}


\end{document}
%%% Local Variables:
%%% mode: latex
%%% TeX-master: t
%%% End:
