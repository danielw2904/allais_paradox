\documentclass{beamer}
\mode<presentation> {

  \usetheme{Boadilla}
  \usecolortheme{dolphin}
  % \setbeamertemplate{footline}
  \setbeamercovered{transparent}

  \AtBeginSection[]{
  \begin{frame}
  \vfill
  \centering
  \begin{beamercolorbox}[sep=8pt,center,shadow=true,rounded=true]{title}
    \usebeamerfont{title}\insertsectionhead\par%
  \end{beamercolorbox}
  \vfill
  \end{frame}
}

}

\usepackage[german]{babel}
\usepackage[style=authoryear, backend=bibtex]{biblatex} 
   \bibliography{bibliography} 
\usepackage{eurosym}
\title{Irrationalit\"at}
   
\author{Stefan Gabler \& Daniel Winkler}

\date{3. Mai 2017}

\begin{document}

\begin{frame}[plain]
  \titlepage
  \textcite{oliver2003quantitative}\\\citetitle{oliver2003quantitative}\\

\end{frame}

\begin{frame}
  \frametitle{\"Uberblick}
  \tableofcontents
\end{frame}

\section{Experiment}
\label{sec:experiment}

\begin{frame}  
 \frametitle{Rationalit\"at unter VWL-Studierenden}
  
 % \begin{columns}
    
 %   \begin{column}{0.5\textwidth}
      
  \begin{block}{1. Entscheidung}
   \begin{itemize}
      \item[A] Sie bekommen \euro 100  mit Sicherheit.
      \item[B] Sie haben eine 10\% Chance auf \euro 500,\\ eine 89\% Chance auf \euro 100\\ und eine 1\% Chance nichts zu bekommen.
     \end{itemize}

   \end{block}

 %  \end{column}      

 % \begin{column}{0.5\textwidth}

    \begin{block}{2. Entscheidung}
      \begin{itemize}
      \item[C] Sie haben eine 11\% Chance auf \euro 100\\und eine 89\% Chance nichts zu gewinnen.
      \item[D] Sie haben eine 10\% Chance auf \euro 500\\und eine 90\% Chance nichts zu gewinnen.
      \end{itemize}
   \end{block}

 %  \end{column}
  
 % \end{columns}
  
 \end{frame}

 \begin{frame}
   \frametitle{Rationalit\"at unter VWL-Studierenden}
   \begin{center}
     

   \huge{\textbf{Warum bevorzugen Sie diese Kombination?}}
   \end{center}
 \end{frame}

\section{Axiome der Erwartungsnutzentheorie}
\label{sec:einfuhrung}
\begin{frame}
  \frametitle{Die Axiome nach Von Neumann \& Morgenstern}
   \begin{enumerate}
  \item<1,4> Unabh"angigkeit von (irrelevanten) Alternativen.\\
    F"ur alle $x, y, z \in$ A\footnote{Menge der Alternative} und alle $\alpha \in (0,1)$:\\
    \begin{equation}
      \label{eq:4}
      x \succ y \Rightarrow \alpha x +(1-\alpha) z \succ \alpha y + (1-\alpha) z  \tag{Unabh"angigkeit}
    \end{equation}

        
  \item<2> F"ur alle $x, y, z \in$ A:
      \begin{equation}
      \label{eq:1}
     x \succ y \succ z \Rightarrow x \succ z \tag{Transitivit"at}
    \end{equation}
    F\"ur alle x, y $\in$ A gilt eine und nur eine der folgenden Relationen:
    \begin{equation}
      \label{eq:2}
     x \succ y \text{ oder } y \succ x \text{ oder } x \sim y \tag{Vollst"andigkeit  \& Asymmetrie}
    \end{equation}
  \item<3> Es existieren ein $\alpha \in [0,1]$, sodass wenn $x \succ y \succ z$:
    \begin{equation}
      \label{eq:3}
      y \sim \alpha x + (1-\alpha) z \tag{Stetigkeit}
    \end{equation}
  \end{enumerate}

\end{frame}

\section{Allais-Paradoxon}

\begin{frame}
  \frametitle{Beispiel \textcite[S. 527]{allais_paradox}}
    \begin{itemize}

  \item \textbf{Bevorzugen Sie Situation A oder Situation B?}\\
  \textit{Situation A:} Sie bekommen 100 Mio.\footnote{In Franken.} mit Sicherheit.\\
  \textit{Situation B:} 10\% Chance 500 Mio. zu bekommen. 89\% Chance 100 Mio. zu bekommen. 1\% Chance nichts zu bekommen. 
  
\item \textbf{Bevorzugen Sie Situation C oder Situation D?}\\
  \textit{Situation C:} 11\% Chance 100 Mio. zu bekommen. 89\% Chance nichts zu bekommen.\\
  \textit{Situation D:} 10\% Chance 500 Mio. zu bekommen. 90\% Chance nichts zu bekommen.
\end{itemize}

\end{frame}

\begin{frame}
  \frametitle{Vorhersage aufgrund des Unabh\"angigkeitsaxioms}
  \begin{itemize}
  \item<1-> Falls der erwartete Nutzen aus A gr\"o"ser ist als jener aus B \begin{equation}\label{eq:6}
     A \succ  B \Leftrightarrow u(100) > 0.10* u(500) + \alert<3>{0.89* u(100)} \tag{Situation A}
   \end{equation}
   
 \item<2-> ... dann ist der erwartete Nutzen aus C gr\"o"ser als jener aus D \begin{equation}
     \label{eq:7}
   \Rightarrow \alert<3>{0.11*u(100)} > 0.10*u(500) \Leftrightarrow C \succ D \tag{Situation B}
 \end{equation}
 
\item<3-> ... denn es wird einfach $0.89*u(100)$ auf beiden Seiten abgezogen.
  \end{itemize}
\end{frame}

\begin{frame}
  \frametitle{Allais-Paradoxon}
  \begin{itemize}
  \item Viele bevorzugen A \& D\\~\\
  \item<2-> M\"ogliche Begr\"undungen:
    \begin{itemize}
    \item Komplette Sicherheit in A wird bevorzugt
    \item Abneigung gegen\"uber "`Verlust"' 
    \item Erwartete reue
    \item Fehlensch\"atzung von Wahrscheinlichkeiten 
    \end{itemize}
  \end{itemize}
\end{frame}

\section{Allais-Paradoxon im Kontext von Gesundheit}

\section{Fragen \& Diskussion}

\begin{frame}
  \frametitle{Fragen \& Diskussion}
  \begin{itemize}
  \item<1> Gibt es noch Fragen?\\~\\
  \item<2-> Finden Sie, dass das Allais-Paradoxon "`Irrationalit\"at"' aufzeigt?\\~\\
  \item<2-> Ist damit die Annahme der Rationalit\"at in der VWL zu verwerfen?\\
    F\"ur interessierte siehe Diskussion in \textcite[Kapitel 12]{gintis2009bounds}.\\~\\
  \item<2-> W\"urden Sie nun beim Experiment anders antworten?\\~\\
  \item<2-> Welchen Unterschied macht der Kontext Geld/Gesundheit?\\~\\
    
  \end{itemize}
\end{frame}


\begin{frame}[allowframebreaks]
  \frametitle{Literatur}
  \nocite{barbera2004handbook}
  \nocite{gintis2009bounds}
  \nocite{rieck2012spieltheorie}
  \nocite{osborne2004introduction}
  \printbibliography
\end{frame}
\end{document}
%%% Local Variables:
%%% mode: latex
%%% TeX-master: t
%%% End:
